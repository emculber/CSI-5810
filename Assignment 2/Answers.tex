\documentclass{report}
\usepackage[showframe=false]{geometry}
\usepackage{titlesec}
\usepackage{amsmath}
\usepackage{graphicx}
\usepackage{float}

\pagenumbering{gobble}

\geometry{tmargin=60pt,bmargin=90pt,lmargin=90pt,
rmargin=90pt}

\titleformat{\chapter}{\normalfont\huge}{\thechapter.}{20pt}{\huge}
\titlespacing*{\chapter} {0pt}{0pt}{10pt}

\begin{document}

\chapter{Section 1}

\section{a}

\par 
Precision and Recall have and inverse relationship given the number  of documenets retrieved. Therefore if there was an increase in the number of documnets that were returned in the queries then the percision and recall values would change with that increase. percision would most likely drop while recall would most likely increase.

\section{b}

20 Documents
80% precision
50% recall
A = Number of relevant records retrieved
B = number of relevant documents that are not retrieved by the system

Recall = A/(A+B)

\chapter{Section 2}

\chapter{Section 3}

\chapter{Section 4}

\chapter{Section 5}

\end{document}







































